% Options for packages loaded elsewhere
\PassOptionsToPackage{unicode}{hyperref}
\PassOptionsToPackage{hyphens}{url}
\documentclass[
]{article}
\usepackage{xcolor}
\usepackage[margin=1in]{geometry}
\usepackage{amsmath,amssymb}
\setcounter{secnumdepth}{-\maxdimen} % remove section numbering
\usepackage{iftex}
\ifPDFTeX
  \usepackage[T1]{fontenc}
  \usepackage[utf8]{inputenc}
  \usepackage{textcomp} % provide euro and other symbols
\else % if luatex or xetex
  \usepackage{unicode-math} % this also loads fontspec
  \defaultfontfeatures{Scale=MatchLowercase}
  \defaultfontfeatures[\rmfamily]{Ligatures=TeX,Scale=1}
\fi
\usepackage{lmodern}
\ifPDFTeX\else
  % xetex/luatex font selection
\fi
% Use upquote if available, for straight quotes in verbatim environments
\IfFileExists{upquote.sty}{\usepackage{upquote}}{}
\IfFileExists{microtype.sty}{% use microtype if available
  \usepackage[]{microtype}
  \UseMicrotypeSet[protrusion]{basicmath} % disable protrusion for tt fonts
}{}
\makeatletter
\@ifundefined{KOMAClassName}{% if non-KOMA class
  \IfFileExists{parskip.sty}{%
    \usepackage{parskip}
  }{% else
    \setlength{\parindent}{0pt}
    \setlength{\parskip}{6pt plus 2pt minus 1pt}}
}{% if KOMA class
  \KOMAoptions{parskip=half}}
\makeatother
\usepackage{graphicx}
\makeatletter
\newsavebox\pandoc@box
\newcommand*\pandocbounded[1]{% scales image to fit in text height/width
  \sbox\pandoc@box{#1}%
  \Gscale@div\@tempa{\textheight}{\dimexpr\ht\pandoc@box+\dp\pandoc@box\relax}%
  \Gscale@div\@tempb{\linewidth}{\wd\pandoc@box}%
  \ifdim\@tempb\p@<\@tempa\p@\let\@tempa\@tempb\fi% select the smaller of both
  \ifdim\@tempa\p@<\p@\scalebox{\@tempa}{\usebox\pandoc@box}%
  \else\usebox{\pandoc@box}%
  \fi%
}
% Set default figure placement to htbp
\def\fps@figure{htbp}
\makeatother
\setlength{\emergencystretch}{3em} % prevent overfull lines
\providecommand{\tightlist}{%
  \setlength{\itemsep}{0pt}\setlength{\parskip}{0pt}}
\usepackage{bookmark}
\IfFileExists{xurl.sty}{\usepackage{xurl}}{} % add URL line breaks if available
\urlstyle{same}
\hypersetup{
  pdftitle={Estudo sobre crimes},
  pdfauthor={Felipe Silveira},
  hidelinks,
  pdfcreator={LaTeX via pandoc}}

\title{Estudo sobre crimes}
\author{Felipe Silveira}
\date{2025-11-02}

\begin{document}
\maketitle

\section{Instalar pacotes}\label{instalar-pacotes}

library(tidyverse) library(dplyr) library(ggplot2)

\section{Carregando os dados}\label{carregando-os-dados}

dados\_seguranca \textless-
read\_csv(`/Users/felipesilveira/RStudio/Estudo sobre
crimes/br\_fbsp\_absp\_municipio.csv')

\section{Verificando estrutura}\label{verificando-estrutura}

glimpse(dados\_seguranca)

\section{Primeiras linhas}\label{primeiras-linhas}

head(dados\_seguranca)

\section{Resumo estatístico}\label{resumo-estatuxedstico}

summary(dados\_seguranca)

\section{Transformando algumas colunas em tipo
factor}\label{transformando-algumas-colunas-em-tipo-factor}

dados\_limpos \textless- dados\_seguranca \textbar\textgreater{} mutate(
ano = as.factor(ano), sigla\_uf = as.factor(sigla\_uf), grupo =
as.factor(grupo) )

\subsection{EDA}\label{eda}

\section{1. Como as mortes violentas evoluíram ao longo dos anos no
Brasil?}\label{como-as-mortes-violentas-evoluuxedram-ao-longo-dos-anos-no-brasil}

\section{Agrupar por ano e somar mortes
violentas}\label{agrupar-por-ano-e-somar-mortes-violentas}

evolucao\_brasil \textless- dados\_limpos \textbar\textgreater{}
group\_by(ano) \textbar\textgreater{} summarise(
total\_mortes\_violentas =
sum(quantidade\_mortes\_violentas\_intencionais, na.rm = TRUE),
total\_homicidios = sum(quantidade\_homicidio\_doloso, na.rm = TRUE) )

print(evolucao\_brasil)

\section{Visualização de mortes
violentas}\label{visualizauxe7uxe3o-de-mortes-violentas}

ggplot(evolucao\_brasil, aes(x = ano, y = total\_mortes\_violentas,
group = 1)) + geom\_line(color = ``red'', size = 1.5) +
geom\_point(color = ``red'', size = 3) + labs( title = ``Evolução das
mortes violentas intencionais no Brasil(Municípios da amostra)'', x =
``Ano'', y = ``Quantidade total'' ) + theme\_minimal()

\section{2. Quais estados tiveram mais homicídios dolosos no último ano
registrado?}\label{quais-estados-tiveram-mais-homicuxeddios-dolosos-no-uxfaltimo-ano-registrado}

\section{Encontrar último ano}\label{encontrar-uxfaltimo-ano}

ultimo\_ano \textless- max(as.numeric(as.character(dados\_limpos\$ano)))

\section{Filtrar pelo ano, agrupar por UF e
ordenar}\label{filtrar-pelo-ano-agrupar-por-uf-e-ordenar}

ranking\_uf \textless- dados\_limpos \textbar\textgreater{} filter(ano
== ultimo\_ano) \textbar\textgreater{} group\_by(sigla\_uf)
\textbar\textgreater{} summarise( total\_homicidios =
sum(quantidade\_homicidio\_doloso, na.rm = TRUE) )
\textbar\textgreater{} arrange(desc(total\_homicidios))

print(ranking\_uf)

\section{Gráfico de barras com os 10
primeiros}\label{gruxe1fico-de-barras-com-os-10-primeiros}

ggplot(head(ranking\_uf, 10), aes(x = reorder(sigla\_uf,
total\_homicidios), y = total\_homicidios)) + geom\_bar(stat =
``identity'', fill = ``steelblue'') + coord\_flip() + labs( title =
paste(``Top 10 UFs por homicídio doloso em'', ultimo\_ano), x = ``UF'',
y = ``Quantidade total de homicídios'' ) + theme\_bw()

\#Existe relação entr roubo de veículos e tráfico de entorpecentes? \#
Selecionar colunas e remover NAs dados\_correlacao \textless-
dados\_limpos \textbar\textgreater{} select(
quantidade\_roubo\_furto\_veiculos, quantidade\_trafico\_entorpecente )
\textbar\textgreater{} na.omit()

\section{Calcular a correlação}\label{calcular-a-correlauxe7uxe3o}

cor(dados\_correlacao\(quantidade_roubo_furto_veiculos, dados_correlacao\)quantidade\_trafico\_entorpecente)

\section{Visualizar a correlação com um gráfico de dispersão
(scatterplot)}\label{visualizar-a-correlauxe7uxe3o-com-um-gruxe1fico-de-dispersuxe3o-scatterplot}

ggplot(dados\_correlacao, aes(x = quantidade\_roubo\_furto\_veiculos, y
= quantidade\_trafico\_entorpecente)) + geom\_point(alpha = 0.5) + \#
alpha = transparência geom\_smooth(method = ``lm'', color = ``blue'') +
\# Adiciona uma linha de tendência labs( title = ``Correlação entre
Roubo/Furto de Veículos e Tráfico'', x = ``Roubo e Furto de Veículos'',
y = ``Tráfico de Entorpecentes'' ) + theme\_light()

\section{Total de feminicidios}\label{total-de-feminicidios}

total\_feminicidios \textless- dados\_limpos \textbar\textgreater{}
summarise( total\_geral = sum(quantidade\_feminicidio, na.rm = TRUE) )

print(total\_feminicidios)

\section{Total por estado}\label{total-por-estado}

\section{Quantidade de feminicidios por
ano}\label{quantidade-de-feminicidios-por-ano}

feminicidios\_por\_ano \textless- dados\_limpos \textbar\textgreater{}
group\_by(ano) \textbar\textgreater{} summarise( total\_anual =
sum(quantidade\_feminicidio, na.rm = TRUE) ) \textbar\textgreater{}
arrange(ano)

\section{Visualizando em gráficos}\label{visualizando-em-gruxe1ficos}

ggplot(feminicidios\_por\_ano, aes(x = as.factor(ano), y = total\_anual,
group = 1)) + geom\_line(color = ``purple'', size = 1.2) +
geom\_point(color = ``purple'', size = 2.5) + labs( title = ``Evolução
dos Feminicídios por Ano (na amostra)'', x = ``Ano'', y = ``Quantidade
Total de Feminicídios'' ) + theme\_minimal()

\end{document}
